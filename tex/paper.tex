% v2-acmsmall-sample.tex, dated March 6 2012
% This is a sample file for ACM small trim journals
%
% Compilation using 'acmsmall.cls' - version 1.3 (March 2012), Aptara Inc.
% (c) 2010 Association for Computing Machinery (ACM)
%
% Questions/Suggestions/Feedback should be addressed to => "acmtexsupport@aptaracorp.com".
% Users can also go through the FAQs available on the journal's submission webpage.
%
% Steps to compile: latex, bibtex, latex latex
%
% For tracking purposes => this is v1.3 - March 2012

\documentclass[prodmode,acmtecs]{acmsmall} % Aptara syntax

% Package to generate and customize Algorithm as per ACM style
\usepackage[ruled]{algorithm2e}
\usepackage{listings}
\renewcommand{\algorithmcfname}{ALGORITHM}
\SetAlFnt{\small}
\SetAlCapFnt{\small}
\SetAlCapNameFnt{\small}
\SetAlCapHSkip{0pt}
\IncMargin{-\parindent}

% Copyright
%\setcopyright{acmcopyright}
%\setcopyright{acmlicensed}
%\setcopyright{rightsretained}
%\setcopyright{usgov}
%\setcopyright{usgovmixed}
%\setcopyright{cagov}
%\setcopyright{cagovmixed}

% Document starts
\begin{document}

% Title portion
\title{Reverse-Mode Automatic Differentiation in Haskell Using the Accelerate Library}
\author{JAMES BRADBURY
\affil{Stanford University}
FARHAN KATHAWALA
\affil{Stanford University}}
% NOTE! Affiliations placed here should be for the institution where the
%       BULK of the research was done. If the author has gone to a new
%       institution, before publication, the (above) affiliation should NOT be changed.
%       The authors 'current' address may be given in the "Author's addresses:" block (below).
%       So for example, Mr. Abdelzaher, the bulk of the research was done at UIUC, and he is
%       currently affiliated with NASA.

\begin{abstract}
Automatic Differentiation is a method for applying differentiation strategies to source
code, by taking a computer program and deriving from that program a separate program which
calculates the derivatives of the output of the first program. Because of this, Automatic
Differentiation is of vital importance to most deep learning tasks as it allows
for the easy backpropogation of complex calculations in order to minimize the loss of a
learning algorithm. Most of these calculations are some variant of composed matrix and vector
operations (such as matrix multiplication or dot products), and because Haskell, and the
functional programming paradigm as a whole, stresses the easy composition of pure functions,
the task of providing a good Automatic Differentiation implementation for deep learning
tasks becomes much easier when using Haskell. Our work provides a functioning Automatic
Differentiation library for use with the Accelerate library which implements matrix
operations.
\end{abstract}

%

% We no longer use \terms command
%\terms{Design, Algorithms, Performance}

\keywords{Deep Learning, NVIDIA, CUDA, CUBLAS, Abstract Syntax Trees}

\maketitle


\section{Introduction}

Andreas Griewak in 1989 noted that \cite{griewank1989}

% quote
\begin{quote}
``Reverse Accumulation yields truncation error free gradient values at less than 5 / \emph{n} times the computing time of divided differences.
\end{quote}

Thus, we have known for a long time that Automatic Differentiation is by far a better
approach than calculating out complex derivatives by hand and applying this derivative to
each scalar function whenever our learning computation function changes because Automatic
Differentiation is error-free and takes less time than manually computing the gradient might
take.

Approaches to Automatic Differentiation exist in many forms, the most well known of which is
the Theano library in the Python programming language \cite{bastien2012}, but these
implementations often require workarounds in order to do things not supported by the
underlying programming language. For example, the lack of lazily evaluated functions (also
called thunks) in Python, requires the Theano library to undergo extra compilation steps
in order to have a sufficiently quick differentiation strategy.

Haskell provides a number of features suited to the task at hand such as

\paragraph{Lazy Evalutation}

The ability to chain together many functions and delay their computation is critical as it
necessitates the creation of and ability to traverse a computation graph. This graph of the
computations of a given program is what we utilize to perform automatic differentiation on
any program of any variation of matrix and vector operation composition.

\paragraph{Tools For Transforming a Computation Graph}

Haskell provides Generalized Abstract Data Types (GADTs) as a language extension, which
allow you to create more powerful data types, and as such this allows for the easy creation
of Abstract Syntax Trees which can easily be made into computations.

\paragraph{Available Libraries}

The Accelerate library implements an Abstract Syntax Tree type for all matrix operations
which we can leverage to traverse the computation graph and from that easily compose a
differentiated computation graph. Using other libraries in the Haskell ecosystem, Accelerate
is able to then send this entire computation graph as one kernel to the GPU of a system for
quick calculation. These atomic calculations defined by the GPU allow differentiation of
matrix operations to be done cleanly and quickly.

\smallskip

Utilizing these features we are able to present a small step towards a unified deep-learning
framwork in Haskell. Using the Accelerate library to leverage matrix and vector operations,
we provide the backpropogation ability and future work may be able to combine these elments
into a larger package for one-stop deep learning capabilities.

\section{Implementation}

Our implementation can be broken down into the following steps

\subsection{Binding to CUBLAS}

GPUs allow for atomic matrix and array operations which can be implemented in the Accelerate
library by utilizing the GPU of a system with the CUDA architecture. To implement such
functions as matrix multiplication (called "gemm") and vector addition (called "axpy"), it
was necessary to interface with the CUDA version of the Basic Linear Algebra System (called
CUBLAS). Using the Foreign Function interface of the Accelerate we were able to directly
call CUBLAS functions as long as we implemented a pure, functional implementation of the
function for use when a supported GPU is not present on a system \cite{clifton2014}.
This also required keeping track of device pointers and GPU execution streams,
so below we show a cleaned-up, with some details omitted for brevity, example of the gemv (matrix-vector multiplication) implementation we used

\smallskip
\lstset{language=Haskell}
\begin{lstlisting}{frame=single}
pureGemv :: (Matr,Vect) -> Vect
pureGemv (arr,brr)  = slice result (lift $ Z:.All:.0)
  where
    result          = (fold (+) 0 $ zipWith (*) arrRepl brrRepl)
    bLen            = length brr
    Z :. rowsA :. _ = unlift (shape arr) ::Z:.Exp Int:.Exp Int

    arrRepl         = replicate (lift $ Z:.All:.1:.All) arr
    brrMat          = reshape   (lift $ Z:.bLen:.1) brr
    brrMatT         = transpose brrMat
    brrRepl         = replicate (lift $ Z:.rowsA:. All:.All) brrMatT

cudaGemvF :: (Matr, Vect) -> CIO (Vect)
cudaGemvF (a,b) = do
  let Z :. ra :. ca = arrayShape a   -- m k
      Z :. rb       = arrayShape b   -- k n
  c <- allocateArray $ Z :. ra -- m n
  withDevicePtrs a ms $ \aptr -> do
    withDevicePtrs b ms $ \bptr -> do
      withDevicePtrs c ms $ \cptr -> do
        liftIO $ BL.gemm 1 ra rb 1 bptr0 1 aptr0 ca 0 cptr0 1
        return c
        
  where bptr0 = castDevPtr bptr
 	 	aptr0 = castDevPtr aptr
 	 	cptr0 = castDevPtr cptr
 	 	

gemv :: (Matr, Vect) -> Vect
gemv (v1,v2) = foreignAcc cudaGemv pureGemv $ lift (v1,v2)
  where cudaGemv = CUDAForeignAcc "cudaGemvF"
\end{lstlisting}
\smallskip

\subsection{Retrieving the Abstract Syntax Tree}

Once we can provide a few matrix and vector operations, the user will specify a program
which composes these calculations on some set of inputs with a specific input to be
optimized, in most cases of deep learning algorithms this input is the weight vector.
Given this user-defined program, we need to retrieve the Abstract Syntax Tree representation
of this program in order to perform Automatic Differentiation on the program. The Accelerate
library uses the Generalized Abstract Data Type (GADT) extension to the Haskell language
to create Abstract Syntax Trees in Higher-Order-Abstract-Syntax (HOAS)
\cite{chakravarty2011}. This use of GADTs and Abstract Syntax Trees in HOAS makes it possible
to traverse the tree using pattern matching for all the possible computations and
expressions.

\subsection{Calculating the Differentiation}

To do this, we need to ask the question "What are we taking the derivative with respect to?".
In any kind of machine learning algorithm, we are taking the derivative of the input which
we want to improve, so as to improve the learned computation. This means that we want to take
the derivative with respsect to the weight vector/matrix of the computation. With this in
mind, we traverse the user-defined program's Abstract Syntax Tree and tag the weight
parameters as we come across them. Then, we create out own Abstract Syntax Tree in a similar
method to the one used by the Accelerate library, but our tree starts with the derivative
of the root of the user-defined tree and reverses its way to the top. One can consider
that if a computation in the original tree looks like

\begin{equation}
\label{eqn:01}
y = f(x)
\end{equation}

then obviously the correlated computation in the derived tree would be

\begin{equation}
\label{eqn:02}
dx = f'(dy)
\end{equation}

Similarly, we can think about the derivatives of matrix/vector operations easily, using
Haskell as such. In this manner, the following function

\begin{equation}
\label{eqn:03}
z = \textbf{dot} \; x \; y
\end{equation}

can be derived as

\begin{equation}
\label{eqn:04}
(dx,dy) = (\textbf{map} \; (dz \, *) \; y, \textbf{map} \; (dz \, *) \; x)
\end{equation}

We go through the tree making conversions as such and composing them together until we run
into a function which has as one of its inputs any of the weight inputs we tagged earlier.
If we encounter such a function, then we store the entire graph up to and including that
derivative to evaluate later because, just as the input for Eq. 1 was \emph{x}
and the output of Eq. 2 was \emph{dx}, so too the output of the stored graph will be the
derivative of the tagged weight input. Doing this for the whole graph completes the
Automatic Differentiation.

\section{Summary}

In short, the library we produced can be seen as the first stepping-stone towards bridging
the similar paradigms of functional programming and deep learning. By taking advantage of
function composability, GADTs, Abstract Syntax Trees, and the lazy evaluation offered by
Haskell, we were able to provide a limited, albeit sufficient for some types of deep
learning, library for Automatic Differentiation of any arbitrary, varied composition of
matrix and vector operations such as matrix-vector multiplication, vector-matrix
multiplication, outer product, dot product, and vector addition. Future work in cleaning up
and fully implementing these functions may allow us to release this library as a package
compatible with the Haskell Accelerate library.

% Bibliography
\bibliographystyle{ACM-Reference-Format-Journals}
\bibliography{paper}
                             % Sample .bib file with references that match those in
                             % the 'Specifications Document (V1.5)' as well containing
                             % 'legacy' bibs and bibs with 'alternate codings'.
                             % Gerry Murray - March 2012

\end{document}
% End of paper.tex


